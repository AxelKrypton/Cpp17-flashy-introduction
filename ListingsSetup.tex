\usepackage{listings}
\usepackage{lstautogobble}

\def\transpPerc{100}
\lstdefinestyle{MyCpp}{
breakatwhitespace=false,            % sets if automatic breaks should only happen at whitespace
breaklines=true,                    % sets automatic line breaking
captionpos=b,                       % sets the caption-position to bottom
deletekeywords={...},               % if you want to delete keywords from the given language
escapeinside={@|}{|@},                % if you want to add LaTeX within your code
extendedchars=true,                 % lets you use non-ASCII characters; for 8-bits encodings only,
                                    % does not work with UTF-8
frame=none  ,                       % adds a frame around the code
numbers=none,                       % where to put the line-numbers; possible values are (none, left, right)
numbersep=5pt,                      % how far the line-numbers are from the code
numberstyle=\tiny\color{black},     % the style that is used for the line-numbers
rulecolor=\color{black},            % if not set, the frame-color may be changed on line-breaks within not-black text
                                    % (e.g. comments (green here))
showspaces=false,                   % show spaces everywhere adding particular underscores; it overrides 'showstringspaces'
showstringspaces=false,             % underline spaces within strings only
showtabs=false,                     % show tabs within strings adding particular underscores
stepnumber=2,                       % the step between two line-numbers. If it's 1, each line will be numbered
stringstyle=\color{OliveGreen},     % string literal style
tabsize=2,                          % sets default tabsize to 2 spaces
title=\lstname,                     % show the filename of files included with \lstinputlisting; also try caption instead of title
%
%Base style for this presentation
keepspaces=true,                    % keeps spaces in text, useful for keeping indentation of code
                                    % (possibly needs columns=flexible)
keywordstyle=\color{Cyan},          % keyword style
language=C++,
basicstyle=\ttfamily\scriptsize\color{black},
keywordstyle=\color{OliveGreen},
stringstyle=\color{ForestGreen},
commentstyle=\color{red},
moredelim=[is][\color{ForestGreen}]{|+}{+|},
moredelim=[is][\color{black}]{|=}{=|},
literate=% literate={<replace>}{<replacement text>}{<width>}
  {\#define}{{{\color{CarnationPink}\#define}}}{6}
  {\#if}{{{\color{CarnationPink}\#if}}}{3}
  {\#include}{{{\color{CarnationPink}\#include}}}{7},
morekeywords={size_t, T, pair, tuple, array, unordered_map, string, map, Widget,
              BadWidget, is_integral, initializer_list, byte, Type, logic_error,
              error_info, variant, optional, any, string_view, filesystem, list, byte,
              stringstream, char_traits, bad_variant_access, var_t, bad_any_cast, path,
              sequenced_policy, parallel_policy, parallel_unsequenced_policy, vector, C,
              mdspan, expected},
emph=[1]{main, get_mix, type_name, make_tuple, status, find, end, message, zip, boolalpha, size,
         isIntegral,  foo, bar, is_success, do_something, do_something_ref, my_callback, is_error,
         splice, invoke, apply, get, visit, index, in_place_type, to_int, eof, slice, value_or,
         substr, min, find_first_not_of, remove_prefix, getSV, what, make_optional, any_cast, make_any,
         copy_file, create_directory, append, exists, file_size, space, push_back, find,
         for_each, begin, uncaught_exception, uncaught_exceptions, print},
emphstyle=[1]{\color{NavyBlue}}, %Functions
emph=[2]{x, y, z, cout, uns, n, pi, letter, label, mapping, key, content, argc, argv, it, m, s,
         gadget, x2, x3, x4, b1, b2, b3, b4, ei, err, message, var, v, w, i, oi, stm, number, a, b,
         str, start, finish, nullopt, bad, e, filePath, fileSize, tmpPath, par, longVector, result,
         obj, obj2, vec, arg},
emphstyle=[2]{\color{Orange}}, %Variables
emph=[3]{if, else, while, do, for, case, switch, inline, try, catch, goto},
emphstyle=[3]{\color{violet}}, %Loops, if, etc.
emph=[4]{const, auto, break, continue, default, static, return, struct, NULL, sizeof, typedef, using, this,
         template, typename, true, false, throw, public, class, constexpr, decltype, namespace, enum},
emphstyle=[4]{\color{ProcessBlue}}, %Logical keywords
emph=[5]{std, boost, literals, fs, execution},
emphstyle=[5]{\color{Maroon}}, %Namespaces
emph=[6]{__FUNCTION__, __has_include, static_assert, assert},
emphstyle=[6]{\color{Gray}}, %Macros
emph=[7]{second, value, id, count, available},
emphstyle=[7]{\color{Peach!50!Purple}}, %Members
emph=[8]{OK, Bad, type_A, type_B},
emphstyle=[8]{\color{Blue}}, % Enum members
emph=[9]{nodiscard, maybe_unused, fallthrough},
emphstyle=[9]{\color{Goldenrod!50!black}}, %Attributes
emph=[10]{ns1, ns2, ns3},
emphstyle=[10]{\color{MediumSeaGreen}}, %Attributes
% Further functionality
autogobble=true, % lstautogobble needed!
}


\def\CPP{\lstinline[style=MyCpp, basicstyle=\ttfamily\color{black}]}
\lstnewenvironment{Cpp}[1][]
    {\lstset{style=MyCpp, belowskip=-7mm, aboveskip=0pt,#1}}
    {}

\makeatletter
\newenvironment{CenteredBox}{%
\begin{Sbox}}{% Save the content in a box
\end{Sbox}\centerline{\parbox{\wd\@Sbox}{\TheSbox}}}% And output it centered
\makeatother

% =====> Copied from https://raw.githubusercontent.com/AxelKrypton/Bash_lecture_2020/master/TeX/InitialMaterial/ListingsSetupSlides.tex
%Colors for listings
\colorlet{background-color}{gray!20}
\colorlet{basic-color}{black}
\colorlet{keywords-color}{Goldenrod}
\colorlet{comment-color}{red!95!black}
\colorlet{strings-color}{ForestGreen}
\colorlet{builtins-color}{MediumBlue!90!black}
\colorlet{functions-color}{NavyBlue}
\colorlet{variables-color}{DarkOrange}
\colorlet{environment-color}{Gray}
\colorlet{external-color}{SteelBlue}

% https://tex.stackexchange.com/a/34000
\makeatletter
\lst@Key{countblanklines}{true}[t]%
{\lstKV@SetIf{#1}\lst@ifcountblanklines}

\lst@AddToHook{OnEmptyLine}{%
    \lst@ifnumberblanklines\else%
    \lst@ifcountblanklines\else%
    \advance\c@lstnumber-\@ne\relax%
    \fi%
    \fi}
\makeatother

\lstdefinestyle{MyBash}{
    backgroundcolor=\color{background-color}, % choose the background color; you must add \usepackage{color} or \usepackage{xcolor}
    breakatwhitespace=false,            % sets if automatic breaks should only happen at whitespace
    breaklines=true,                    % sets automatic line breaking
    captionpos=b,                       % sets the caption-position to bottom
    deletekeywords={...},               % if you want to delete keywords from the given language
    escapeinside={@|}{|@},              % if you want to add LaTeX within your code
    extendedchars=true,                 % lets you use non-ASCII characters; for 8-bits encodings only,
    % does not work with UTF-8
    frame=single,                       % adds a frame around the code
    framerule=0pt,                      % Width of the frame rule
    framesep=3pt,                       % separation around text
    linewidth=\textwidth,               % defines the base line width for listings
    xleftmargin=6mm,                    % Margin left
    xrightmargin=6mm,                   % Margin right
    numbers=left,                       % where to put the line-numbers; possible values are (none, left, right)
    numberblanklines=false,             % suppress numbers on empty lines
    countblanklines=false,              % NOT standard! Avoid counting empty lines: https://tex.stackexchange.com/a/34000
    numbersep=8pt,                      % how far the line-numbers are from the code
    numberstyle=\tiny\color{black},     % the style that is used for the line-numbers
    rulecolor=\color{black},            % if not set, the frame-color may be changed on line-breaks within not-black text
    % (e.g. comments (green here))
    showspaces=false,                   % show spaces everywhere adding particular underscores; it overrides 'showstringspaces'
    showstringspaces=false,             % underline spaces within strings only
    showtabs=false,                     % show tabs within strings adding particular underscores
    stepnumber=1,                       % the step between two line-numbers. If it's 1, each line will be numbered
    tabsize=2,                          % sets default tabsize to 2 spaces
    title=\lstname,                     % show the filename of files included with \lstinputlisting; also try caption instead of title
    %
    %Base style for this presentation
    keepspaces=true,                    % keeps spaces in text, useful for keeping indentation of code
    % (possibly needs columns=flexible)
    language=bash,
    basicstyle=\ttfamily\scriptsize\color{basic-color},
    keywordstyle=\color{keywords-color},
    stringstyle=\color{strings-color},
    commentstyle=\color{comment-color},
    morestring=[b][\color{strings-color}]{"},
    morestring=[d][\color{strings-color}]{'},
    moredelim=[is][\color{basic-color}]{|+}{+|}, % I will use this for terminal output
    literate={`}{\textasciigrave}1, % https://tex.stackexchange.com/a/466224/128737
    literate={~}{{\textasciitilde}}1,
    % literate=% literate={<replace>}{<replacement text>}{<width>}
    %   {\#define}{{{\color{CarnationPink}\#define}}}{6}
    %   {\#include}{{{\color{CarnationPink}\#include}}}{7},
    alsoletter=0123456789![]/\{\}.:+, % This to mark the symbols in keyword/emph[5] to be highlighted (otherkeywords does not work i.e. it highlights also in comments!) -> manual at page 45
    morekeywords={if, then, else, elif, fi, case, esac, for, select, while, until, do, done, in, function, time, [[, ]], \{, \}, !, coproc}, %https://askubuntu.com/a/513712
    emph=[1]{},
    emphstyle=[1]{\color{functions-color}}, %Functions
    emph=[2]{br},
    emphstyle=[2]{\color{variables-color}}, %Variables
    emph=[4]{PATH, SHELL, IFS, BASH_ALIASES, BASH_REMATCH, PS3, REPLY, HOME, LANGUAGE, EDITOR, PIPESTATUS, PWD, FUNCNEST,
        DIRSTACK, PWD, OLDPWD, SHELLOPTS, BASHOPTS, TIMEFORMAT, COMP_CWORD, COMP_LINE, COMP_POINT, COMP_TYPE, COMP_KEY,
        COMP_WORDBREAKS, COMP_WORDS, COMPREPLY, INPUTRC},
    emphstyle=[4]{\color{environment-color}}, %Environment variables
    emph=[5]{alias, bg, bind, break, builtin, cd, command, compgen, complete, continue, declare, dirs, disown, echo, enable, eval,
        exec, exit, export, false, fc, fg, getopts, hash, help, history, jobs, kill, let, local, logout, popd, printf, pushd, pwd,
        read, readonly, return, set, shift, shopt, source, suspend, test, times, trap, true, type, typeset, ulimit, umask,
        % case, if, until, while  % <--- these built-in are keywords and I leave them highlighted as such
        unalias, unset, wait, :, ., [, ]},
    emphstyle=[5]{\color{builtins-color}}, %Shell built-in
    emph=[6]{man, apropos, ls, rm, g++, chmod, cp, awk, sed, cut, perl, args, date, grep, sleep, tput, seq, cat, wc, sort, uniq, tail,
        head, sdiff, tar, mktemp, mkdir, ps, emacs, systemd, timeout, parallel, xargs, gnuplot, pdflatex, vi, ping, bash,
        egrep, shuf, stat, find, fgrep, bc, tr, paste, expr, diff, touch, git},
    emphstyle=[6]{\color{external-color}}, %(External) commands
    emph=[7]{},
    emphstyle=[7]{\color{variables-color}}, %Class for local variables (usually with bad names)
    emph=[8]{},
    emphstyle=[8]{\color{builtins-color}}, %Class for local commands (usually with bad names)
    %
    %Additional customizations
    belowskip=-7mm,
    aboveskip=3pt,
    autogobble=true, % lstautogobble needed!
}

\lstnewenvironment{Bash}[1][]
  {\lstset{style=MyBash, belowskip=-7mm, aboveskip=0pt,#1}}
  {}
