%-------------------------------%
%  Author: Alessandro Sciarra   %
%    Date: 15 Jun 2022          %
%-------------------------------%

%~~~~~~~~~~~~~~~~~~~~~~~~~~~~~~~~~~~~~~~~~~~~%
\begin{frame}[fragile]{C++17 new library features I will \textbf{NOT} cover}
    \hspace*{5mm}
    \begin{tabular}{l}
        \URL[PB]{https://en.cppreference.com/w/cpp/numeric/special_functions}{~Mathematical special functions} (still  \URL[PT]{https://en.cppreference.com/w/cpp/compiler_support/17\#C.2B.2B17_library_features}{GNU compiler only})\\[1mm]
        \URL[PB]{https://en.cppreference.com/w/cpp/types/byte}{~\CPP|std::byte|}\\[1mm]
        \URL[PB]{https://en.cppreference.com/w/cpp/utility/functional/invoke}{~\CPP|std::invoke|}\Remark{More understandable \URL[PB]{https://stackoverflow.com/q/43680182/14967071}{SO question}}\\[1mm]
        \URL[PB]{https://en.cppreference.com/w/cpp/utility/apply}{~\CPP|std::apply|} $\;\to\;$ {\footnotesize\URL[PB]{https://stackoverflow.com/q/52449163/14967071}{What's the difference with \CPP|std::invoke|\,?}}\\[1mm]
        \URL[PB]{https://wg21.link/P0083R3}{~Splicing maps and sets} \Remark{i.e. effectively transfer elements \`a la \CPP|std::list::splice|}

    \end{tabular}
\end{frame}
%~~~~~~~~~~~~~~~~~~~~~~~~~~~~~~~~~~~~~~~~~~~~%
\begin{frame}{C++17 new library features I will cover}
    \vspace{-3mm}
    \hspace*{1cm}
    \begin{minipage}[t][0.65\textheight]{\textwidth}
        \tableofcontents
    \end{minipage}
\end{frame}
%============================================%

%============================================%
\section{Throwing in a destructor}
%~~~~~~~~~~~~~~~~~~~~~~~~~~~~~~~~~~~~~~~~~~~~%
\begin{frame}[fragile]{\insertsectionhead: Is it still bad practice?}
    \begin{itemize}
        \item It has been considered bad practice for years\\
        \then \URL[ALERT]{http://www.gotw.ca/gotw/047.htm}{Blog from Herb Sutter (2009)}
        \item \CPP|bool std::uncaught_exception()|\\
        detects only if the current thread has a live exception object\\
        \then it fails to cover all use cases! \Remark{Deprecated in C++17, removed in C++20}
        \item C++17 introduces \CPP|int std::uncaught_exceptions()|\\
        that says how many uncaught exceptions exist\\
        \then see \URL[PB]{https://stackoverflow.com/a/49503194/14967071}{this SO answer} for a nice example
    \end{itemize}
    \begin{varblock}{}[0.68\textwidth]{If you need it}
        You can now safely throw from a destructor!
    \end{varblock}
\end{frame}
%============================================%

%============================================%
\section{\CPP|std::string\_view|}
%~~~~~~~~~~~~~~~~~~~~~~~~~~~~~~~~~~~~~~~~~~~~%
\begin{frame}[fragile]{\insertsectionhead}
    \vspace{-3mm}
    \begin{itemize}
        \item A \alert{non-owning} reference to a string, it is just a view!
        \item Useful for providing an abstraction on top of strings \Remark{e.g.\ for parsing}
        \item There are view modification functions which do not affect the viewed string
        \item \CPP|std::string_view| \alert{doesn't use null terminators} to mark the end of the string
        \item \PP{Prefer passing string views by value to constant refs to strings!}\\
              \then \URL[PB]{https://stackoverflow.com/a/40129198/14967071}{How is it faster?}
        \item More information: \URL[PB]{https://en.cppreference.com/w/cpp/string/basic_string_view}{C++ reference}
    \end{itemize}
\end{frame}
%~~~~~~~~~~~~~~~~~~~~~~~~~~~~~~~~~~~~~~~~~~~~%
\begin{frame}[fragile]{}
    \begin{varblock}{example}[\textwidth]{Basic usage}<only@1>
        \begin{Cpp}
            #include |+<iostream>+|
            #include |+<string>+|
            #include |+<string_view>+|

            int main()
            {
                std::string str {"   trim me"};
                std::string_view v {str};
                v.remove_prefix(std::min(
                                  v.find_first_not_of(" "), v.size()));
                std::cout << '_' << str << "_\t_" << v << "_\n";

                using namespace std::literals;
                auto getSV = [](){ // Noooooo! -> view is NOT OWNING!
                    std::string x = "local";
                    return std::string_view{x};
                };
                std::string_view bad = getSV();
                std::cout << bad << '\n'; // UNDEFINED BEHAVIOUR
            }
        \end{Cpp}
        \begin{Bash}[numbers=none]
            |+_   trim me_    _trim me_
            p4Ek+|
        \end{Bash}
    \end{varblock}
\end{frame}
%============================================%

%============================================%
\section{\CPP|std::variant|}
%~~~~~~~~~~~~~~~~~~~~~~~~~~~~~~~~~~~~~~~~~~~~%
\begin{frame}[fragile]{\insertsectionhead}
    \vspace{-3mm}
    \begin{itemize}
        \item A class template that represents a type-safe union
        \item An instance of \,\CPP|std::variant|\, at any given time either holds a value of one of its alternative types, or in the case of error, no value \Remark{generally hard to achieve}
        \item Assign values at construction or using the \CPP|operator=|
        \item Retrieve values using the \,\CPP|std::get<T>()|\, function
        \item \alert{Revolution in inheritance patterns (e.g. visitor)}
        \item Use the \URL[PB]{https://en.cppreference.com/w/cpp/utility/variant/visit}{non-member function} \,\CPP|std::visit|\, to apply a callable to the value hold by variant(s)
        \item More information: \URL[PB]{https://en.cppreference.com/w/cpp/utility/variant}{C++ reference}
    \end{itemize}
\end{frame}
%~~~~~~~~~~~~~~~~~~~~~~~~~~~~~~~~~~~~~~~~~~~~%
\begin{frame}[fragile]{}
    \begin{varblock}{example}[\textwidth]{Basic usage (\overlaynumber)}<only@1>
        \begin{Cpp}
            #include |+<variant>+|
            #include |+<string>+|
            #include |+<iostream>+|

            int main()
            {
                std::variant<int, float> v, w;
                v = 42; // v contains int
                int i = std::get<int>(v);
                assert(42 == i); // succeeds

                w = std::get<int>(v);
                w = std::get<0>(v); // same effect as the previous line
                w = v;              // same effect as the previous line

                std::variant<std::string, float> v{
                    std::in_place_type<std::string>, 4, 'A'};
                // initializes 1st alternative with std::string{4, 'A'};
                assert(v.index() == 0);
                std::cout << "v=" << std::get<std::string>(v) << '\n';
            }
        \end{Cpp}
        \begin{Bash}[numbers=none]
            |+v=AAAA+|
        \end{Bash}
    \end{varblock}
    \begin{varblock}{example}[\textwidth]{Basic usage (\overlaynumber)}<only@2>
        \begin{Cpp}
            #include |+<variant>+|
            #include |+<iostream>+|

            int main()
            {
                std::variant<int, float> v, w;
                v = 42;
                w = v;

                // Errors:
                //std::get<double>(v); // no double in [int, float]
                //std::get<3>(v);      // valid index values are 0 and 1

                try {
                    // w contains int, not float: will throw
                    std::get<float>(w);
                }
                catch (const std::bad_variant_access& e) {
                    std::cout << e.what() << '\n';
                }
            }
        \end{Cpp}
        \begin{Bash}[numbers=none]
            |+bad_variant_access+|
        \end{Bash}
    \end{varblock}
    \begin{varblock}{example}[\textwidth]{Basic usage (\overlaynumber)}<only@3>
        \begin{Cpp}
            #include |+<variant>+|
            #include |+<iostream>+|
            #include |+<vector>+|

            using var_t = std::variant<int, long, double, std::string>;

            int main()
            {
                std::vector<var_t> vec = {10, 15l, 1.6, "hello"};
                for (auto& v: vec) {
                    // value-returning visitor, returning another variant
                    var_t w = std::visit([](auto&& arg) -> var_t {
                                             return arg + arg;
                                         }, v);
                    // void visitor, only called for side-effects
                    std::visit([](auto&& arg){std::cout << arg;}, w);
                }
            }
        \end{Cpp}
        \begin{Bash}[numbers=none]
            |+20 30 3.2 hellohello+|
        \end{Bash}
    \end{varblock}
\end{frame}
%============================================%

%============================================%
\section{\CPP|std::optional|}
%~~~~~~~~~~~~~~~~~~~~~~~~~~~~~~~~~~~~~~~~~~~~%
\begin{frame}[fragile]{\insertsectionhead}
    \vspace{-3mm}
    \begin{itemize}
        \item A class template that manages an optional contained value, i.e.\ a value that may or may not be present
        \item A common use case for optional is the return value of a function that may fail
        \item No dynamic memory allocation ever takes place\\
              \then \alert{an optional object models an object, not a pointer}
        \item Possible to contextually convert to \CPP|bool|
        \item \CPP|std::nullopt| is used to indicate optional type with uninitialized state
        \item Use \CPP|std::make_optional<T>| if needed
        \item More information: \URL[PB]{https://en.cppreference.com/w/cpp/utility/optional}{C++ reference}
    \end{itemize}
\end{frame}
%~~~~~~~~~~~~~~~~~~~~~~~~~~~~~~~~~~~~~~~~~~~~%
\begin{frame}[fragile]{}
    \begin{varblock}{example}[\textwidth]{Optional returning value}<only@1>
        \begin{Cpp}
            #include |+<iostream>+|
            #include |+<optional>+|
            #include |+<sstream>+|

            std::optional<int> to_int(const std::string& s) {
                std::optional<int> oi{};
                int i;
                if(std::stringstream stm(s); stm >> i)
                    if(stm.get() == std::char_traits<char>::eof())
                        oi = i;
                return oi;
            }

            int main()
            {
                auto number = to_int("-42");
                std::cout << *number << " == " << number.value() << '\n';
                // *opt is UB if opt does not contain a value!!
            }
        \end{Cpp}
        \begin{Bash}[numbers=none]
            |+-42 == -42 +|
        \end{Bash}
    \end{varblock}
    \begin{varblock}{example}[\textwidth]{Optional arguments}<only@2>
        \begin{Cpp}
            #include |+<iostream>+|
            #include |+<optional>+|
            #include |+<string_view>+|

            auto slice(std::string_view str,
                       std::optional<int> start,
                       std::optional<int> finish)
            {
                auto a = start.value_or(0);
                auto b = end.value_or(str.size());
                return str.substr(a, b-a);
            }

            int main()
            {
                std::cout
                    << slice("Hello world!", 6, std::nullopt) << '\n'
                    << slice("Hello world!", std::nullopt, 5) << '\n';
            }
        \end{Cpp}
        \begin{Bash}[numbers=none]
            |+world!
            Hello+|
        \end{Bash}
    \end{varblock}
\end{frame}
%============================================%

%============================================%
\section{\CPP|std::any|}
%~~~~~~~~~~~~~~~~~~~~~~~~~~~~~~~~~~~~~~~~~~~~%
\begin{frame}[fragile]{\insertsectionhead}
    \vspace{-3mm}
    \begin{itemize}
        \item A type-safe, single-value container\\ for any \textbf{copy-constructible type}.
        \item Use \CPP|operator=| to assign value
        \item Use \CPP|std::any_cast<T>| to retrieve value
        \item Use \CPP|std::make_any<T>| in generic programming
        \item More information: \URL[PB]{}{C++ reference}
    \end{itemize}
\end{frame}
%~~~~~~~~~~~~~~~~~~~~~~~~~~~~~~~~~~~~~~~~~~~~%
\begin{frame}[fragile]{}
    \begin{varblock}{example}[\textwidth]{Basic usage}<only@1>
        \begin{Cpp}
            #include |+<any>+|
            #include |+<iostream>+|

            int main()
            {
                std::any a = 1;
                std::cout << '[' << std::any_cast<int>(a) << ", ";
                a = 3.14;
                std::cout << std::any_cast<double>(a) << ", ";
                a = true;
                std::cout << std::boolalpha
                          << std::any_cast<bool>(a) << "]\n";

                try {
                    std::cout << std::any_cast<float>(a) << '\n';
                }
                catch (const std::bad_any_cast& e) {
                    std::cout << e.what() << '\n';
                }
            }
        \end{Cpp}
        \begin{Bash}[numbers=none]
            |+[1, 3.14, true]
            bad any cast+|
        \end{Bash}
    \end{varblock}
\end{frame}
%============================================%

%============================================%
\section{\CPP|std::filesystem|}
%~~~~~~~~~~~~~~~~~~~~~~~~~~~~~~~~~~~~~~~~~~~~%
\begin{frame}[fragile]{\insertsectionhead}
    \vspace{-3mm}
    \begin{itemize}
        \item This library was originally developed as \URL[PB]{http://www.boost.org/doc/libs/release/libs/filesystem/doc/index.htm}{boost.filesystem}
        \item It provides a standard way to manipulate \textbf{file objects}
              \begin{itemize}
                  \item directory
                  \item regular file
                  \item symbolic link
                  \item other special file types (block, character, fifo, socket)
              \end{itemize}
              \textbf{file names} and \textbf{paths} in a filesystem.
        \item \PP{Many additional features are available!}
        \item More information: \URL[PB]{https://en.cppreference.com/w/cpp/filesystem}{C++ reference}
    \end{itemize}
\end{frame}
%~~~~~~~~~~~~~~~~~~~~~~~~~~~~~~~~~~~~~~~~~~~~%
\begin{frame}[fragile]{}
    \begin{varblock}{example}[\textwidth]{Copy big file if existing and enough memory is available}<only@1>
        \begin{Cpp}
            #include |+<iostream>+|
            #include |+<filesystem>+|

            namespace fs = std::filesystem;

            int main()
            {
                const auto filePath {"bigFileToCopy"};
                if (fs::exists(filePath)) {
                    const auto fileSize {fs::file_size(filePath)};
                    fs::path tmpPath {"/tmp"};
                    if (fs::space(tmpPath).available > fileSize)
                    {
                        fs::create_directory(tmpPath.append("example"));
                        fs::copy_file(filePath, tmpPath.append("new"));
                    }
                } else
                    std::cout
                        << "File \"" << filePath << "\" not found.\n";
            }
        \end{Cpp}
        \begin{Bash}[numbers=none]
            |+File "bigFileToCopy" not found.+|
        \end{Bash}
    \end{varblock}
\end{frame}
%============================================%

%============================================%
\section{Parallel algorithms}
%~~~~~~~~~~~~~~~~~~~~~~~~~~~~~~~~~~~~~~~~~~~~%
\begin{frame}[fragile]{\insertsectionhead}
    \vspace{-3mm}
    \begin{itemize}
        \item Many of the STL algorithms started to support the parallel execution policies of the following possible types:
              \begin{itemize}
                  \item \CPP|sequenced_policy|
                  \item \CPP|parallel_policy|
                  \item \CPP|parallel_unsequenced_policy|
              \end{itemize}
        \item \alert{When using parallel execution policy, it is the programmer's responsibility to avoid data races and deadlocks}
        \item More information: \URL[PB]{https://en.cppreference.com/w/cpp/algorithm\#Execution_policies}{C++ reference}
        \item Unfortunately  \URL[PQ]{https://en.cppreference.com/w/cpp/compiler_support/17\#C.2B.2B17_library_features}{(Apple) Clang does not support them yet}
    \end{itemize}
\end{frame}
%~~~~~~~~~~~~~~~~~~~~~~~~~~~~~~~~~~~~~~~~~~~~%
\begin{frame}[fragile]{}
    \begin{varblock}{example}[\textwidth]{Untested code}<only@1>
        \begin{Cpp}
            #include |+<algorithm>+|
            #include |+<vector>+|
            #include |+<execution>+|

            int main()
            {
                int a[] = {0,1};
                std::vector<int> v;
                std::for_each(std::execution::par,
                              std::begin(a), std::end(a),
                              [&](int i){
                                  v.push_back(i*2+1); // Error: data race
                              });

                std::vector<int> longVector /* Initialization */;
                // Find element using parallel execution policy
                auto result = std::find(std::execution::par,
                                        std::begin(longVector),
                                        std::end(longVector),
                                        2);
            }
        \end{Cpp}
    \end{varblock}
\end{frame}
%============================================%


