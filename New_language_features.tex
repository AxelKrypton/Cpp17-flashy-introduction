%-------------------------------%
%  Author: Alessandro Sciarra   %
%    Date: 14 Jun 2022          %
%-------------------------------%

%~~~~~~~~~~~~~~~~~~~~~~~~~~~~~~~~~~~~~~~~~~~~%
\begin{frame}[fragile]{C++17 new language features I will \textbf{not} cover}
    \begin{tabular}{l}
        \URL[PB]{https://en.cppreference.com/w/cpp/language/class_template_argument_deduction}{~Template argument deduction for class templates}\\[1mm]
        \URL[PB]{https://en.cppreference.com/w/cpp/language/template_parameters}{~Declaring non-type template parameters with auto}\Remark{see item (4) at link}\\[1mm]
        \URL[PB]{https://en.cppreference.com/w/cpp/language/fold}{~Folding expressions}\Remark{relevant for variadic templates}\\[1mm]
        \URL[PB]{https://en.cppreference.com/w/cpp/language/lambda}{~\CPP|constexpr| lambda}\Remark{see specifiers at link}\\[1mm]
        \URL[PB]{https://en.cppreference.com/w/cpp/language/lambda\#Lambda_capture}{~Lambda capture this by value}\Remark{see item (8) at link}\\[1mm]
        \URL[PB]{https://en.cppreference.com/w/cpp/language/character_literal}{~UTF-8 character literals}\Remark{see item (2) at link}\\[1mm]
        \URL[PB]{https://en.cppreference.com/w/cpp/preprocessor/include}{\CPP|__has_include|}\Remark{see items (4) and (5) at link}\\
    \end{tabular}
\end{frame}
%~~~~~~~~~~~~~~~~~~~~~~~~~~~~~~~~~~~~~~~~~~~~%
\begin{frame}{C++17 new language features I will cover}
    \tableofcontents
\end{frame}
%============================================%

%============================================%
\section{Structured bindings}
%~~~~~~~~~~~~~~~~~~~~~~~~~~~~~~~~~~~~~~~~~~~~%
\begin{frame}[fragile]{\insertsectionhead}
    \vspace{-3mm}
    \begin{itemize}
        \item De-structuring initialization that allows writing\\
              \quad\CPP|auto [ x, y, z ] = expression;|
        \item The type of \CPP|expression| is a tuple-like object, whose elements are bound to the variables \CPP|x|, \CPP|y|, and \CPP|z| which this construct declares
        \item Tuple-like objects can be\\
              \begin{itemize}
                  \item[] \CPP|std::pair|,
                  \item[] \CPP|std::tuple|,
                  \item[] \CPP|std::array|,
              \end{itemize}
              and aggregate structures
        \item More information: \URL[PB]{https://en.cppreference.com/w/cpp/language/structured_binding}{C++ reference}
    \end{itemize}
\end{frame}
%~~~~~~~~~~~~~~~~~~~~~~~~~~~~~~~~~~~~~~~~~~~~%
\begin{frame}[fragile]{}
    \PrepareURLsymbol[red]
    \begin{varblock}{example}[\textwidth]{One of my favourite features}<only@1>
        \begin{Cpp}
            #include |+<iostream>+|
            #include |+<utility>+|
            #include |+<string_view>+| // @|\URL*[red]{https://stackoverflow.com/a/56766138}{Very cool idea}|@
            template <typename T> constexpr auto type_name() {...}

            auto get_mix()
            {
                return std::make_tuple(42U, 42, 3.14, 'b', "Hello");
            }

            int main(){
                auto [uns, n, pi, letter, label] = get_mix();
                std::cout << "[" << type_name<decltype(uns)>() << ", "
                          << type_name<decltype(n)>() << ", "
                          << type_name<decltype(pi)>() << ", "
                          << type_name<decltype(letter)>() << ", "
                          << type_name<decltype(label)>() << "]\n";
            }
        \end{Cpp}
        \smallskip
        \begin{Bash}[numbers=none]
            |+[unsigned int, int, double, char, const char *]+|
        \end{Bash}
    \end{varblock}
    \begin{varblock}{example}[\textwidth]{Even cooler}<only@2>
        \begin{Cpp}
            #include |+<iostream>+|
            #include |+<unordered_map>+|

            int main(){
                std::unordered_map<std::string, int> mapping {
                    {"a", 1},
                    {"b", 2},
                    {"c", 3}
                };

                // Destructure by reference.
                for (const auto& [key, content] : mapping) {
                    std::cout << key << " -> " << content << "\n";
                }
            }
        \end{Cpp}
        \smallskip
        \begin{Bash}[numbers=none]
            |+b -> 2
            c -> 3
            a -> 1+|
        \end{Bash}
    \end{varblock}
\end{frame}
%============================================%

%============================================%
\section{Selection statements with initializer}
%~~~~~~~~~~~~~~~~~~~~~~~~~~~~~~~~~~~~~~~~~~~~%
\begin{frame}[fragile]{\insertsectionhead}
    \vspace{-3mm}
    \begin{itemize}
        \item New versions of the \CPP|if| and \CPP|switch| statements which simplify common code patterns and help keeping scopes tight.\\
              \begin{varblock}{}[0.73\textwidth]{How they look like:}
                  \small
                  \CPP|if(init-statement; condition)|\\[1mm]
                  \CPP|switch(init-statement; condition)|
              \end{varblock}
              \par\medskip
        \item Names declared by the \CPP|init-statement| (if it is a declaration) and names declared by \CPP|condition| (if it is a declaration) are in the same scope of all branches.
    \end{itemize}
\end{frame}
%~~~~~~~~~~~~~~~~~~~~~~~~~~~~~~~~~~~~~~~~~~~~%
\begin{frame}[fragile]{}
    \begin{varblock}{example}[\textwidth]{New if and switch feature}<only@1>
        \begin{Cpp}
            #include |+<iostream>+|
            #include |+<map>+|
            #include |+<iomanip>+|
            #include |+"widget.hpp"+|

            int main(int argc, char *argv[])
            {
                std::map<int, std::string> m;
                if (auto it = m.find(10); it != m.end())
                    std::cout << it->second.size();
                else
                    std::cout << std::boolalpha << (it==m.end()) << "\n";

                switch (Widget gadget(argc, argv);
                        auto s = gadget.status())
                {
                    case OK: gadget.zip(); break;
                    case Bad: throw BadWidget(s.message());
                }
            }
        \end{Cpp}
        \begin{Bash}[numbers=none]
            |+true+|
        \end{Bash}
    \end{varblock}
\end{frame}
%============================================%

%============================================%
\section{Compile time if conditions}
%~~~~~~~~~~~~~~~~~~~~~~~~~~~~~~~~~~~~~~~~~~~~%
\begin{frame}[fragile]{\insertsectionhead}
    \vspace{-3mm}
    \begin{itemize}
        \item In a \CPP|constexpr if| statement, the value of the condition must be a \URL[PB]{https://en.cppreference.com/w/cpp/language/constant_expression\#Converted_constant_expression}{contextually converted constant expression of type \CPP|bool|} \Remark{until C++23}
        \item  Depending on the condition, either the \CPP|if| or the \CPP|else| statement is discarded.
        \item The return statements in a discarded statement do not participate in function return type deduction
        \item Outside a template, a discarded statement is fully checked\\
              {\footnotesize$\to\;$\CPP|if constexpr| is not a substitute for the \CPP|#if| preprocessing directive}
        \item More information: \URL[PB]{https://en.cppreference.com/w/cpp/language/if\#Constexpr_if}{C++ reference}
    \end{itemize}
\end{frame}
%~~~~~~~~~~~~~~~~~~~~~~~~~~~~~~~~~~~~~~~~~~~~%
\begin{frame}[fragile]{}
    \begin{varblock}{example}[\textwidth]{Compile time if clause}<only@1>
        \begin{Cpp}
            template <typename T>
            constexpr bool isIntegral() {
                if constexpr (std::is_integral<T>::value) {
                    return true;
                } else {
                    return false;
                }
            }
            struct Widget {};
            int main(){
                if constexpr(false) {
                    int n = 0;
                    //int *p = n; // Error even though discarded
                }
                static_assert(isIntegral<char>() == true);
                static_assert(isIntegral<double>() == false);
                static_assert(isIntegral<Widget>() == true);
            }
        \end{Cpp}
        \begin{Bash}[numbers=none]
            |+$ clang++ -std=c++17 Example.cpp -o Example
            Constexpr-if.cpp:26:5: error: static_assert failed due to requirement 'isIntegral<Widget>() == true'
            [...]+|
        \end{Bash}
    \end{varblock}
\end{frame}
%============================================%

%============================================%
\section{Inline variables}
%~~~~~~~~~~~~~~~~~~~~~~~~~~~~~~~~~~~~~~~~~~~~%
\begin{frame}[fragile]{\insertsectionhead}
    \vspace{-3mm}
    \begin{itemize}
        \item An inline variable has the same semantics as an inline function.\\[1mm]
              {\scriptsize Together with other properties
                There may be more than one definition of an inline variable in the program as long as each definition appears in a different translation unit and (for non-static inline variables) all definitions are identical. For example, an inline variable may be defined in a header file that is included in multiple source files.\par}
        \item More information: \URL[PB]{https://en.cppreference.com/w/cpp/language/inline}{C++ reference}
    \end{itemize}
    \begin{varblock}{example}[\textwidth]{Define a static member variable in \textbf{header} file!}<only@1>
        \begin{Cpp}
            struct Widget {
                Widget() : id{count++} {}
                ~Widget() { count--; }
                int id;
                // declare and initialize count to 0 within the class
                static inline int count{0};
            };
        \end{Cpp}
    \end{varblock}
\end{frame}
%============================================%

%============================================%
\section{New rules for auto deduction from braced-init-list}
%~~~~~~~~~~~~~~~~~~~~~~~~~~~~~~~~~~~~~~~~~~~~%
\begin{frame}[fragile]{\insertsectionhead}
    \begin{itemize}
        \item Till C++14$^\star$, in
              \begin{center}
                  \CPP|auto x{3};|
              \end{center}
              \CPP|x| is deduced to be a
              \begin{center}
                  \CPP|std::initializer_list<int>|
              \end{center}
              which is at least unexpected \Remark{if not misleading}
        \item In C++17 it is now deduced to be an \CPP|int|
        \item Direct-list-initialization from a multiple-element braced-init-list is now ill-formed
    \end{itemize}
    \PrepareURLsymbol[PB]
    \FrameRemark{$^\star$More precisely till paper \URL*{http://www.open-std.org/jtc1/sc22/wg21/docs/papers/2014/n3922.html}{N3922} approved in 2014 already in Urbana by the core working group (CWG) in \URL*{https://www.open-std.org/jtc1/sc22/wg21/docs/papers/2014/n4251.html}{motion 16}.}
\end{frame}
%~~~~~~~~~~~~~~~~~~~~~~~~~~~~~~~~~~~~~~~~~~~~%
\begin{frame}[fragile]{}
    \PrepareURLsymbol[red]
    \begin{varblock}{example}[\textwidth]{Pretty logic and easy to remember now}<only@1>
        \begin{Cpp}
            #include |+<iostream>+|
            #include |+<utility>+|
            #include |+<string_view>+| // @|\URL*[red]{https://stackoverflow.com/a/56766138}{Very cool idea}|@
            template <typename T> constexpr auto type_name() {...}

            int main(){
                //auto x1 {1, 2, 3}; // error: not a single element
                auto x2 = {1, 2, 3};
                auto x3 {3u};
                auto x4 {3.0f};
                std::cout << type_name<decltype(x2)>() << "\n"
                          << type_name<decltype(x3)>() << "\n"
                          << type_name<decltype(x4)>() << "\n";
            }
        \end{Cpp}
        \begin{Bash}[numbers=none]
            |+std::initializer_list<int>
            unsigned int
            float+|
        \end{Bash}
    \end{varblock}
\end{frame}
%============================================%

%============================================%
\section{Nested namespaces}
%~~~~~~~~~~~~~~~~~~~~~~~~~~~~~~~~~~~~~~~~~~~~%
\begin{frame}[fragile]{\insertsectionhead}
    \vspace{-6mm}
    \begin{varblock}{example}[\textwidth]{Reducing boilerplate}<only@1>
        \begin{Cpp}
            // No alternatives till C++14
            namespace A {
                namespace B {
                    namespace C {
                        int i;
                    }
                }
            }

            // The code above can be written like this:
            namespace A::B::C {
                int i;
            }
        \end{Cpp}
    \end{varblock}
\end{frame}
%============================================%

%============================================%
\section{Direct-list-initialization of enums}
%~~~~~~~~~~~~~~~~~~~~~~~~~~~~~~~~~~~~~~~~~~~~%
\begin{frame}[fragile]{\insertsectionhead}
    \vspace{-3mm}
    An enumeration can be initialized from an integer without a cast, using list initialization, if all of the following are true:
    \begin{itemize}
        \small
        \item the initialization is direct-list-initialization
        \item the initializer list has only a single element
        \item the enumeration is either scoped or unscoped with underlying type fixed
        \item the conversion is non-narrowing
    \end{itemize}
    \begin{varblock}{example}[\textwidth]{}<only@1>
        \begin{Cpp}
            enum byte : unsigned char {};
            byte b1 {0};         // OK
            byte b2 {-1};        // ERROR: narrowing conversion
            byte b3 = byte{1};   // OK
            byte b4 = byte{256}; // ERROR: narrowing conversion

            enum Type {type_A, type_B};
            Type x{1};           // ERROR: underline type not fixed
        \end{Cpp}
    \end{varblock}
\end{frame}
%============================================%

%============================================%
\section{New attributes}
%~~~~~~~~~~~~~~~~~~~~~~~~~~~~~~~~~~~~~~~~~~~~%
\begin{frame}[fragile]{\insertsectionhead}
    \vspace{-3mm}
    \begin{tabular}{p{\textwidth}}
        \CPP|[[fallthrough]]|
             \URL[PB]{https://en.cppreference.com/w/cpp/language/attributes/fallthrough}{C++ reference}\\[1mm]
             It indicates to the compiler that falling through in a \CPP|switch| statement is intended behaviour\\[3mm]
        \CPP|[[nodiscard]]|
             \URL[PB]{https://en.cppreference.com/w/cpp/language/attributes/nodiscard}{C++ reference}\\[1mm]
             It encourages the compiler to issue a warning if a function declared nodiscard or a function returning an enumeration or class declared nodiscard by value is called from a discarded-value expression other than a cast to \CPP|void|\\[3mm]
        \CPP|[[maybe_unused]]|
             \URL[PB]{https://en.cppreference.com/w/cpp/language/attributes/maybe_unused}{C++ reference}\\[1mm]
             It suppresses compiler warnings on unused entities\\
    \end{tabular}
\end{frame}
%~~~~~~~~~~~~~~~~~~~~~~~~~~~~~~~~~~~~~~~~~~~~%
\begin{frame}[fragile]{}
    \begin{varblock}{example}[\textwidth]{Fall through cases}<only@1>
        \begin{Cpp}
            switch (n) {
                case 1:
                case 2:
                    foo();
                    [[fallthrough]];
                case 3: // no warning on fallthrough
                    bar();
                case 4: // compiler may warn on fallthrough
                    if(n<3)
                    {
                        foo();
                        [[fallthrough]];
                    }
                    else
                        return;
                default:
                    throw std::logic_error("Hit default!");
            }
        \end{Cpp}
    \end{varblock}
    \begin{varblock}{example}[\textwidth]{Warn when discarding nodiscard types returned by value}<only@2>
        \begin{Cpp}
            // Only issues a warning when returned by value.
            struct [[nodiscard]] error_info {
                int id{0};
            };

            error_info do_something() {
                return error_info{};
            }

            error_info& do_something_ref() {
                static error_info err{};
                return err;
            }

            int main(){
                do_something();      // Warning
                do_something_ref();  // No warning!
            }
        \end{Cpp}
        \begin{Bash}[numbers=none]
            |+Attributes.cpp:16:5: warning: ignoring return value of function declared with 'nodiscard' attribute [...]+|
        \end{Bash}
    \end{varblock}
    \begin{onlyenv}<3>
        \begin{varblock}{example}[\textwidth]{Warn when discarding return values}
            \begin{Cpp}
                [[nodiscard]] bool do_something() {
                    return is_success();
                }

                do_something(); // warning: ignoring return value
                // declared with attribute 'nodiscard'
            \end{Cpp}
        \end{varblock}
        \bigskip
        \begin{varblock}{example}[\textwidth]{Unused entities}
            \begin{Cpp}
                void my_callback(std::string message,
                                 [[maybe_unused]] bool is_error) {
                    // Don't care if message is an error message
                    log(message);
                }
            \end{Cpp}
        \end{varblock}
    \end{onlyenv}
\end{frame}
%============================================%

%============================================%
\section{Throwing in a destructor}
%~~~~~~~~~~~~~~~~~~~~~~~~~~~~~~~~~~~~~~~~~~~~%
\begin{frame}[fragile]{\insertsectionhead: Is it still bad practice?}
    \begin{itemize}
        \item It has been considered bad practice for years\\
              \then \URL[ALERT]{http://www.gotw.ca/gotw/047.htm}{Blog from Herb Sutter (2009)}
        \item \CPP|bool std::uncaught_exception()|\\
              detects only if the current thread has a live exception object\\
              \then it fails to cover all use cases!
        \item C++17 introduces \CPP|int std::uncaught_exceptions()|\\
              that says how many uncaught exceptions exist\\
              \then see \URL[PB]{https://stackoverflow.com/a/49503194/14967071}{this SO answer} for a nice example
    \end{itemize}
\end{frame}
%============================================%

%============================================%
\section{Mandatory elision of copy/move operations}
%~~~~~~~~~~~~~~~~~~~~~~~~~~~~~~~~~~~~~~~~~~~~%
\begin{frame}{\insertsectionhead}
    In return and assignment statements, often$^{\star}$ {\footnotesize\quad\URL[PB]{https://en.cppreference.com/w/cpp/language/copy_elision}{[Precise rule here]}}
    \begin{itemize}
        \item The compilers are required to omit the copy and move construction of class objects
        \item Even if the copy/move constructor and the destructor have observable side-effects
        \item The objects are constructed directly into the storage where they would otherwise be copied/moved to \item The copy/move constructors need not be present or accessible
    \end{itemize}
    \begin{varblock}{alert}[\textwidth]{When copy/move operations are not guaranteed}
        Even if the copy/move constructor is not called, it still must be present and accessible, otherwise the program is ill-formed!
    \end{varblock}
    \PrepareURLsymbol[PB]
    \FrameRemark{$^{\star}$ \textbf{often} $\equiv$ when the operand or initializer expression is a \URL*{https://en.cppreference.com/w/cpp/language/value_category}{prvalue} of the same class type (ignoring cv-qualification).}
\end{frame}
%~~~~~~~~~~~~~~~~~~~~~~~~~~~~~~~~~~~~~~~~~~~~%
\begin{frame}[fragile]{}
    \begin{varblock}{example}[\textwidth]{}<only@1>
        \begin{Cpp}
            #include |+<iostream>+|

            struct C {
                C() { std::cout << "Default ctor called.\n"; }
                C(const C&) { std::cout << "Copy ctor called.\n"; }
                C(C&&) { std::cout << "Move ctor called.\n"; }
            };

            C foo() { return C(); } // Guaranteed to perform copy elision
            C bar() { C c; return c; } // Maybe performs copy elision

            int main() {
                C obj = foo(); // Move constructor isn't called
                std::cout << "-------\n";
                C obj2 = bar();
            }
        \end{Cpp}
        \begin{lstlisting}[style=MyBash, numbers=none]
            |+$ g++ -fno-elide-constructors -std=c++17\
                  -o Copy-move-elision Copy-move-elision.cpp\
                  && ./Copy-move-elision
            Default ctor called.
            -------
            Default ctor called.
            Move ctor called.+|
        \end{lstlisting}
    \end{varblock}
    \begin{varblock}{example}[\textwidth]{}<only@2>
        \begin{Cpp}
            #include |+<iostream>+|

            struct C {
                C() { std::cout << "Default ctor called.\n"; }
                C(const C&) = delete;
                C(C&&) = delete;
            };

            C foo() { return C(); } // Guaranteed to perform copy elision

            /*
             * Uncommenting results in compilation error, since
             * move/copy elision is not guaranteed here!
             */
            // C bar() { C c; return c; }

            int main() {
                C obj = foo(); //Move constructor isn't called
            }
        \end{Cpp}
        \begin{lstlisting}[style=MyBash, numbers=none]
            |+$ g++ -fno-elide-constructors -std=c++17\
                  -o Copy-move-elision Copy-move-elision.cpp\
                  && ./Copy-move-elision
            Default ctor called.+|
        \end{lstlisting}
    \end{varblock}
    \begin{varblock}{example}[\textwidth]{}<only@3>
        \begin{Cpp}
            #include |+<iostream>+|

            struct C {
                C() { std::cout << "Default ctor called.\n"; }
                C(const C&) = delete;
                C(C&&) = delete;
            };

            C bar() { C c; return c; }

            int main() { return 0; }
        \end{Cpp}
        \begin{lstlisting}[style=MyBash, numbers=none]
            |+$ g++ -std=c++17\
                  -o Copy-move-elision Copy-move-elision.cpp\
                  && ./Copy-move-elision
            Copy-move-elision.cpp:9:23: error: call to deleted constructor of 'C'
            C bar() { C c; return c; }
            ^
            Copy-move-elision.cpp:5:5: note: 'C' has been explicitly marked deleted here
            C(const C&) = delete;
            ^
            1 error generated.+|
        \end{lstlisting}
    \end{varblock}
\end{frame}
%============================================%

