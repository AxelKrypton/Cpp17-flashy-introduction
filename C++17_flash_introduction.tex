\documentclass[usenames,dvipsnames,svgnames,14pt]{beamer}
\usepackage[english]{babel}
\usepackage{fontspec,fontawesome,mathabx,fp,pgfornament}
\usepackage{TeXnicalities}

\setsansfont{Yanone Kaffeesatz}[
    UprightFont     = *-Regular ,
    BoldFont        = *-Bold ,
    BoldItalicFont  = *-Bold ,
    BoldSlantedFont = *-Bold ,
    ItalicFont      = *-Light ,
    SlantedFont     = *-Light ,
    SmallCapsFont   = *-Thin
]
\graphicspath{{Figures/}}

\mode<presentation>
{
    \usetheme{Z02}
    \setbeamersize{text margin left=8mm, text margin left=8mm}
}

\usetikzlibrary{
    positioning,
    shapes,
}
\usepackage{listings}
\usepackage{lstautogobble}

\def\transpPerc{100}
\lstdefinestyle{MyCpp}{
breakatwhitespace=false,            % sets if automatic breaks should only happen at whitespace
breaklines=true,                    % sets automatic line breaking
captionpos=b,                       % sets the caption-position to bottom
deletekeywords={...},               % if you want to delete keywords from the given language
escapeinside={@|}{|@},                % if you want to add LaTeX within your code
extendedchars=true,                 % lets you use non-ASCII characters; for 8-bits encodings only,
                                    % does not work with UTF-8
frame=none  ,                       % adds a frame around the code
numbers=none,                       % where to put the line-numbers; possible values are (none, left, right)
numbersep=5pt,                      % how far the line-numbers are from the code
numberstyle=\tiny\color{black},     % the style that is used for the line-numbers
rulecolor=\color{black},            % if not set, the frame-color may be changed on line-breaks within not-black text
                                    % (e.g. comments (green here))
showspaces=false,                   % show spaces everywhere adding particular underscores; it overrides 'showstringspaces'
showstringspaces=false,             % underline spaces within strings only
showtabs=false,                     % show tabs within strings adding particular underscores
stepnumber=2,                       % the step between two line-numbers. If it's 1, each line will be numbered
stringstyle=\color{OliveGreen},     % string literal style
tabsize=2,                          % sets default tabsize to 2 spaces
title=\lstname,                     % show the filename of files included with \lstinputlisting; also try caption instead of title
%
%Base style for this presentation
keepspaces=true,                    % keeps spaces in text, useful for keeping indentation of code
                                    % (possibly needs columns=flexible)
keywordstyle=\color{Cyan},          % keyword style
language=C++,
basicstyle=\ttfamily\scriptsize\color{black},
keywordstyle=\color{OliveGreen},
stringstyle=\color{ForestGreen},
commentstyle=\color{red},
moredelim=[is][\color{ForestGreen}]{|+}{+|},
moredelim=[is][\color{black}]{|=}{=|},
literate=% literate={<replace>}{<replacement text>}{<width>}
  {\#define}{{{\color{CarnationPink}\#define}}}{6}
  {\#if}{{{\color{CarnationPink}\#if}}}{3}
  {\#include}{{{\color{CarnationPink}\#include}}}{7},
morekeywords={size_t, T, pair, tuple, array, unordered_map, string, map, Widget,
              BadWidget, is_integral, initializer_list, byte, Type, logic_error,
              error_info, variant, optional, any, string_view, filesystem, list, byte,
              stringstream, char_traits, bad_variant_access, var_t, bad_any_cast, path,
              sequenced_policy, parallel_policy, parallel_unsequenced_policy, vector, C,
              mdspan, expected},
emph=[1]{main, get_mix, type_name, make_tuple, status, find, end, message, zip, boolalpha, size,
         isIntegral,  foo, bar, is_success, do_something, do_something_ref, my_callback, is_error,
         splice, invoke, apply, get, visit, index, in_place_type, to_int, eof, slice, value_or,
         substr, min, find_first_not_of, remove_prefix, getSV, what, make_optional, any_cast, make_any,
         copy_file, create_directory, append, exists, file_size, space, push_back, find,
         for_each, begin, uncaught_exception, uncaught_exceptions, print},
emphstyle=[1]{\color{NavyBlue}}, %Functions
emph=[2]{x, y, z, cout, uns, n, pi, letter, label, mapping, key, content, argc, argv, it, m, s,
         gadget, x2, x3, x4, b1, b2, b3, b4, ei, err, message, var, v, w, i, oi, stm, number, a, b,
         str, start, finish, nullopt, bad, e, filePath, fileSize, tmpPath, par, longVector, result,
         obj, obj2, vec, arg},
emphstyle=[2]{\color{Orange}}, %Variables
emph=[3]{if, else, while, do, for, case, switch, inline, try, catch, goto},
emphstyle=[3]{\color{violet}}, %Loops, if, etc.
emph=[4]{const, auto, break, continue, default, static, return, struct, NULL, sizeof, typedef, using, this,
         template, typename, true, false, throw, public, class, constexpr, decltype, namespace, enum},
emphstyle=[4]{\color{ProcessBlue}}, %Logical keywords
emph=[5]{std, boost, literals, fs, execution},
emphstyle=[5]{\color{Maroon}}, %Namespaces
emph=[6]{__FUNCTION__, __has_include, static_assert, assert},
emphstyle=[6]{\color{Gray}}, %Macros
emph=[7]{second, value, id, count, available},
emphstyle=[7]{\color{Peach!50!Purple}}, %Members
emph=[8]{OK, Bad, type_A, type_B},
emphstyle=[8]{\color{Blue}}, % Enum members
emph=[9]{nodiscard, maybe_unused, fallthrough},
emphstyle=[9]{\color{Goldenrod!50!black}}, %Attributes
emph=[10]{ns1, ns2, ns3},
emphstyle=[10]{\color{MediumSeaGreen}}, %Attributes
% Further functionality
autogobble=true, % lstautogobble needed!
}


\def\CPP{\lstinline[style=MyCpp, basicstyle=\ttfamily\color{black}]}
\lstnewenvironment{Cpp}[1][]
    {\lstset{style=MyCpp, belowskip=-7mm, aboveskip=0pt,#1}}
    {}

\makeatletter
\newenvironment{CenteredBox}{%
\begin{Sbox}}{% Save the content in a box
\end{Sbox}\centerline{\parbox{\wd\@Sbox}{\TheSbox}}}% And output it centered
\makeatother

% =====> Copied from https://raw.githubusercontent.com/AxelKrypton/Bash_lecture_2020/master/TeX/InitialMaterial/ListingsSetupSlides.tex
%Colors for listings
\colorlet{background-color}{gray!20}
\colorlet{basic-color}{black}
\colorlet{keywords-color}{Goldenrod}
\colorlet{comment-color}{red!95!black}
\colorlet{strings-color}{ForestGreen}
\colorlet{builtins-color}{MediumBlue!90!black}
\colorlet{functions-color}{NavyBlue}
\colorlet{variables-color}{DarkOrange}
\colorlet{environment-color}{Gray}
\colorlet{external-color}{SteelBlue}

% https://tex.stackexchange.com/a/34000
\makeatletter
\lst@Key{countblanklines}{true}[t]%
{\lstKV@SetIf{#1}\lst@ifcountblanklines}

\lst@AddToHook{OnEmptyLine}{%
    \lst@ifnumberblanklines\else%
    \lst@ifcountblanklines\else%
    \advance\c@lstnumber-\@ne\relax%
    \fi%
    \fi}
\makeatother

\lstdefinestyle{MyBash}{
    backgroundcolor=\color{background-color}, % choose the background color; you must add \usepackage{color} or \usepackage{xcolor}
    breakatwhitespace=false,            % sets if automatic breaks should only happen at whitespace
    breaklines=true,                    % sets automatic line breaking
    captionpos=b,                       % sets the caption-position to bottom
    deletekeywords={...},               % if you want to delete keywords from the given language
    escapeinside={@|}{|@},              % if you want to add LaTeX within your code
    extendedchars=true,                 % lets you use non-ASCII characters; for 8-bits encodings only,
    % does not work with UTF-8
    frame=single,                       % adds a frame around the code
    framerule=0pt,                      % Width of the frame rule
    framesep=3pt,                       % separation around text
    linewidth=\textwidth,               % defines the base line width for listings
    xleftmargin=6mm,                    % Margin left
    xrightmargin=6mm,                   % Margin right
    numbers=left,                       % where to put the line-numbers; possible values are (none, left, right)
    numberblanklines=false,             % suppress numbers on empty lines
    countblanklines=false,              % NOT standard! Avoid counting empty lines: https://tex.stackexchange.com/a/34000
    numbersep=8pt,                      % how far the line-numbers are from the code
    numberstyle=\tiny\color{black},     % the style that is used for the line-numbers
    rulecolor=\color{black},            % if not set, the frame-color may be changed on line-breaks within not-black text
    % (e.g. comments (green here))
    showspaces=false,                   % show spaces everywhere adding particular underscores; it overrides 'showstringspaces'
    showstringspaces=false,             % underline spaces within strings only
    showtabs=false,                     % show tabs within strings adding particular underscores
    stepnumber=1,                       % the step between two line-numbers. If it's 1, each line will be numbered
    tabsize=2,                          % sets default tabsize to 2 spaces
    title=\lstname,                     % show the filename of files included with \lstinputlisting; also try caption instead of title
    %
    %Base style for this presentation
    keepspaces=true,                    % keeps spaces in text, useful for keeping indentation of code
    % (possibly needs columns=flexible)
    language=bash,
    basicstyle=\ttfamily\scriptsize\color{basic-color},
    keywordstyle=\color{keywords-color},
    stringstyle=\color{strings-color},
    commentstyle=\color{comment-color},
    morestring=[b][\color{strings-color}]{"},
    morestring=[d][\color{strings-color}]{'},
    moredelim=[is][\color{basic-color}]{|+}{+|}, % I will use this for terminal output
    literate={`}{\textasciigrave}1, % https://tex.stackexchange.com/a/466224/128737
    literate={~}{{\textasciitilde}}1,
    % literate=% literate={<replace>}{<replacement text>}{<width>}
    %   {\#define}{{{\color{CarnationPink}\#define}}}{6}
    %   {\#include}{{{\color{CarnationPink}\#include}}}{7},
    alsoletter=0123456789![]/\{\}.:+, % This to mark the symbols in keyword/emph[5] to be highlighted (otherkeywords does not work i.e. it highlights also in comments!) -> manual at page 45
    morekeywords={if, then, else, elif, fi, case, esac, for, select, while, until, do, done, in, function, time, [[, ]], \{, \}, !, coproc}, %https://askubuntu.com/a/513712
    emph=[1]{},
    emphstyle=[1]{\color{functions-color}}, %Functions
    emph=[2]{br},
    emphstyle=[2]{\color{variables-color}}, %Variables
    emph=[4]{PATH, SHELL, IFS, BASH_ALIASES, BASH_REMATCH, PS3, REPLY, HOME, LANGUAGE, EDITOR, PIPESTATUS, PWD, FUNCNEST,
        DIRSTACK, PWD, OLDPWD, SHELLOPTS, BASHOPTS, TIMEFORMAT, COMP_CWORD, COMP_LINE, COMP_POINT, COMP_TYPE, COMP_KEY,
        COMP_WORDBREAKS, COMP_WORDS, COMPREPLY, INPUTRC},
    emphstyle=[4]{\color{environment-color}}, %Environment variables
    emph=[5]{alias, bg, bind, break, builtin, cd, command, compgen, complete, continue, declare, dirs, disown, echo, enable, eval,
        exec, exit, export, false, fc, fg, getopts, hash, help, history, jobs, kill, let, local, logout, popd, printf, pushd, pwd,
        read, readonly, return, set, shift, shopt, source, suspend, test, times, trap, true, type, typeset, ulimit, umask,
        % case, if, until, while  % <--- these built-in are keywords and I leave them highlighted as such
        unalias, unset, wait, :, ., [, ]},
    emphstyle=[5]{\color{builtins-color}}, %Shell built-in
    emph=[6]{man, apropos, ls, rm, g++, chmod, cp, awk, sed, cut, perl, args, date, grep, sleep, tput, seq, cat, wc, sort, uniq, tail,
        head, sdiff, tar, mktemp, mkdir, ps, emacs, systemd, timeout, parallel, xargs, gnuplot, pdflatex, vi, ping, bash,
        egrep, shuf, stat, find, fgrep, bc, tr, paste, expr, diff, touch, git},
    emphstyle=[6]{\color{external-color}}, %(External) commands
    emph=[7]{},
    emphstyle=[7]{\color{variables-color}}, %Class for local variables (usually with bad names)
    emph=[8]{},
    emphstyle=[8]{\color{builtins-color}}, %Class for local commands (usually with bad names)
    %
    %Additional customizations
    belowskip=-7mm,
    aboveskip=3pt,
    autogobble=true, % lstautogobble needed!
}

\lstnewenvironment{Bash}[1][]
  {\lstset{style=MyBash, belowskip=-7mm, aboveskip=0pt,#1}}
  {}


%Commands for this presentation
\defbeamertemplate{description item}{bold label}{\textbf{\insertdescriptionitem}}
\makeatletter
    \newcommand*{\overlaynumber}{\number\beamer@slideinframe}
    \newlength{\textsize}
    \setlength{\textsize}{\f@size pt}
    \newcommand{\emoji}[1]{\;\raisebox{-0.15\textsize}{\includegraphics[height=\f@size pt]{#1}}}
\makeatother
\newcommand{\then}{\raisebox{2pt}{$\;\drsh\;$}}
\newcommand{\MakeCoverSlide}{%
    \begin{frame}[plain,noframenumbering]
        \begin{tikzpicture}[remember picture, overlay]
            \usebeamercolor{title page background canvas}
            \fill[bg] (current page.south west) rectangle (current page.north east);
            \node[text depth=0.5ex, anchor=north] (title) at ($(current page.north)-(0mm,8mm)$)
               {\huge\usebeamercolor[fg]{title}\inserttitle};
            \node[below = 3mm of title] (photo)
            {%
                \raisebox{-0.5\height}{\includegraphics[height=0.33\textheight]{C++_Logo}}%
                \hspace{6mm}\raisebox{-0.5\height}{\usebeamercolor[fg]{title}\fontsize{60}{72}\selectfont 17}
            };
            \node[below = 5mm of photo] (author) {\large\insertauthor};
            \node[below = 3mm of author] (date) {\usebeamerfont{date}\insertdate};
            \node[anchor=south, outer sep=0mm, font=\footnotesize] at (current page.south) {
                \includegraphics[width=10mm]{CC-BY}\quad
                \raisebox{1pt}{\color{PB}\href{https://github.com/AxelKrypton}{\faGithub\ AxelKrypton}}
            };
            \node[anchor=south west] at (current page.south west) {
                \includegraphics[width=20mm]{LogoGoethe}
            };
            \node[anchor=south east] at (current page.south east) {
                \includegraphics[width=20mm]{LogoCRC}
            };
        \end{tikzpicture}
    \end{frame}
}

\AtBeginPart{%
    \begin{frame}
        \partpage
    \end{frame}
}

%\AtBeginSection{%
%    \begin{frame}
%        \sectionpage
%    \end{frame}
%}

\begin{document}
    %~~~~~~~~~~~~~~~~~~~~~~~~~~~~~~~~~~~~~~~~~~~~%
    \title{A flashy introduction to}
    \author{Alessandro~Sciarra}
    \date{14 April 2023}
    %~~~~~~~~~~~~~~~~~~~~~~~~~~~~~~~~~~~~~~~~~~~~%
    \MakeCoverSlide
    %~~~~~~~~~~~~~~~~~~~~~~~~~~~~~~~~~~~~~~~~~~~~%
    \begin{frame}[fragile]{Kick-off riddle: Does this code compile?}{\uncover<2->{An (ab)use of the language}}
        \begin{columns}
            \begin{column}{0.58\textwidth}
                \begin{varblock}{}[\textwidth]{}
                    \begin{Cpp}
                        #include |+<iostream>+|

                        int main() {
                            |=https://github.com/AxelKrypton=|
                            int x = 42;
                            std::cout << x << '\n';
                        }
                    \end{Cpp}
                    \begin{uncoverenv}<2->
                        \begin{Bash}[numbers=none]
                            42
                        \end{Bash}
                    \end{uncoverenv}
                \end{varblock}
            \end{column}
            \begin{column}{0.42\textwidth}
                \centering
                \includegraphics[width=\columnwidth]{WIITY}
            \end{column}
        \end{columns}
        \vspace{3mm}
        \begin{uncoverenv}<2->
            \begin{center}
                \Large
                \URL[PB]{https://godbolt.org/z/Y6Yb7M79T}{~Yes, it does!} But why?\\[5mm]
                \footnotesize
                \uncover<3>{
                    \emoji{Sunglasses}\;
                    Because \;\texttt{https:}\; is a \;\CPP|goto|\; label and the rest of the line a comment! \emoji{Sunglasses}
                }
            \end{center}
        \end{uncoverenv}
    \end{frame}
    %~~~~~~~~~~~~~~~~~~~~~~~~~~~~~~~~~~~~~~~~~~~~%
    \begin{frame}{My main expertise for the CRC-TR\,211}{\{\,in terms of programming languages\,\}}
        \begin{tikzpicture}[every node/.style={circle, very thick, draw}, node distance=2mm]
            \node[PS, minimum size=30mm, visible on=<1>] (bash) {Bash};
            \node[PP, minimum size=27mm, right = of bash, alt=<1>{fill=none}{fill=PB!10, fill on=<2>}] (cpp) {C++};
            \begin{scope}[scope on=<1>]
                \node[PT, minimum size=23mm, right = of cpp, yshift=11.5mm] (latex) {LaTeX};
                \node[PB, minimum size=23mm, right = of latex] (git) {Git};
                \node[PQ, minimum size=15mm, below = of $(latex.south)!0.5!(git.south)$ ] (python) {Python};
            \end{scope}
            \node[PB, draw=none, right = of cpp, visible on=<2>, text width=3cm, align=center]{Today we want to explore some modern features!};
        \end{tikzpicture}
        \begin{varblock}{alert}[0.9\textwidth]{Let's have a chat}
            Clean code as common aspect orthogonal to any language!\\
            Come and torture me with all your nagging questions!
        \end{varblock}
    \end{frame}
    %~~~~~~~~~~~~~~~~~~~~~~~~~~~~~~~~~~~~~~~~~~~~%
    \begin{frame}{C++ standards: A simplistic overview}
        \begin{description}[<+(1)->]
            \item[C++98] First standardization\quad
            \PB{C++03} Some polishing
            \item[C++11] \alert{Turning point}\\
                         \then Move semantics, \CPP|auto|, lambdas, \CPP|constexpr|,\\
                         \phantom{\then }smart pointers, type-traits, variadic templates, etc.
            \item[C++14] Small improvements\\
                         \then {better lambdas and better \CPP|constexpr|,\\
                         \phantom{\then }return type deduction, variable templates, etc.}
            \item[C++17] \PS{Larger improvements $\;\to\;$ this talk!}
            \item[C++20] \alert{Turning point}\\
                         \then Concepts, modules, coroutines, ranges, etc.
            \item[C++23] Small improvements $\;\to\;$ mdspan, expected, print, etc.
        \end{description}
    \end{frame}
    %~~~~~~~~~~~~~~~~~~~~~~~~~~~~~~~~~~~~~~~~~~~~%
    \begin{frame}{Is C++ growing more and more?}{How should I be supposed to learn it?}
        \begin{uncoverenv}<2>
            \begin{center}
                \includegraphics[width=0.9\textwidth]{Toolkit}\\[1mm]
                Think of it as a toolkit, you aren't always gonna need everything!
            \end{center}
        \end{uncoverenv}
    \end{frame}
    %~~~~~~~~~~~~~~~~~~~~~~~~~~~~~~~~~~~~~~~~~~~~%
    \begin{frame}{Disclaimer}
        \vspace{-3mm}
        \centering
        \begin{center}
            \hyperlink{P1}{\PB{\large Part 1:} New language features}\\[3mm]
            \hyperlink{P2}{\PB{\large Part 2:} New library features}
        \end{center}
        \begin{varblock}{alert}{That's not all the folks!}
            I just took the main topics which I think it might be interesting to start with and I will either give you a taste of them or simply point you to some reference!
        \end{varblock}
        \bigskip
        \URL[PS]{https://en.cppreference.com/w/cpp/17}{Cppreference of C++17 standard}\\[3mm]
        \URL[PT]{https://www.open-std.org/jtc1/sc22/wg21/docs/papers/2017/n4659.pdf}{Free draft version, 1608 pages}
    \end{frame}
    %~~~~~~~~~~~~~~~~~~~~~~~~~~~~~~~~~~~~~~~~~~~~%
    \label{P1}
    \part{New language features}
    %-------------------------------%
%  Author: Alessandro Sciarra   %
%    Date: 14 Jun 2022          %
%-------------------------------%

%~~~~~~~~~~~~~~~~~~~~~~~~~~~~~~~~~~~~~~~~~~~~%
\begin{frame}[fragile]{C++17 new language features I will \textbf{NOT} cover}
    \hspace*{5mm}
    \begin{tabular}{l}
        \URL[PB]{https://en.cppreference.com/w/cpp/language/class_template_argument_deduction}{~Template argument deduction for class templates}\\[1mm]
        \URL[PB]{https://en.cppreference.com/w/cpp/language/template_parameters}{~Declaring non-type template parameters with auto}\Remark{see item (4) at link}\\[1mm]
        \URL[PB]{https://en.cppreference.com/w/cpp/language/fold}{~Folding expressions}\Remark{relevant for variadic templates}\\[1mm]
        \URL[PB]{https://en.cppreference.com/w/cpp/language/lambda}{~\CPP|constexpr| lambda}\Remark{see specifiers at link}\\[1mm]
        \URL[PB]{https://en.cppreference.com/w/cpp/language/lambda\#Lambda_capture}{~Lambda capture this by value}\Remark{see item (8) at link}\\[1mm]
        \URL[PB]{https://en.cppreference.com/w/cpp/language/character_literal}{~UTF-8 character literals}\Remark{see item (2) at link}\\[1mm]
        \URL[PB]{https://en.cppreference.com/w/cpp/preprocessor/include}{\CPP|__has_include|}\Remark{see items (4) and (5) at link}\\
    \end{tabular}
\end{frame}
%~~~~~~~~~~~~~~~~~~~~~~~~~~~~~~~~~~~~~~~~~~~~%
\AddLinkToTOCfalse
%~~~~~~~~~~~~~~~~~~~~~~~~~~~~~~~~~~~~~~~~~~~~%
\begin{frame}[label=tocI]{C++17 new language features I will cover}
    \vspace{-3mm}
    \hspace*{1cm}
    \begin{minipage}[t][0.75\textheight]{\textwidth}
        \tableofcontents
    \end{minipage}
\end{frame}
%~~~~~~~~~~~~~~~~~~~~~~~~~~~~~~~~~~~~~~~~~~~~%
\def\labelTOC{tocI}
\AddLinkToTOCtrue
%============================================%

%============================================%
\section{Structured bindings}
%~~~~~~~~~~~~~~~~~~~~~~~~~~~~~~~~~~~~~~~~~~~~%
\begin{frame}[fragile]{\insertsectionhead}
    \vspace{-3mm}
    \begin{itemize}
        \item De-structuring initialization that allows writing\\
              \quad\CPP|auto [ x, y, z ] = expression;|
        \item The type of \CPP|expression| is a tuple-like object, whose elements are bound to the variables \CPP|x|, \CPP|y|, and \CPP|z| which this construct declares
        \item Tuple-like objects can be\\
              \begin{itemize}
                  \item[] \CPP|std::pair|,
                  \item[] \CPP|std::tuple|,
                  \item[] \CPP|std::array|,
              \end{itemize}
              and aggregate structures
        \item More information: \URL[PB]{https://en.cppreference.com/w/cpp/language/structured_binding}{C++ reference}
    \end{itemize}
\end{frame}
%~~~~~~~~~~~~~~~~~~~~~~~~~~~~~~~~~~~~~~~~~~~~%
\begin{frame}[fragile]{}
    \PrepareURLsymbol[red]
    \begin{varblock}{example}[\textwidth]{One of my favourite features}<only@1>
        \begin{Cpp}
            #include |+<iostream>+|
            #include |+<utility>+|
            #include |+<string_view>+| // @|\URL*[red]{https://stackoverflow.com/a/56766138}{Very cool idea}|@
            template <typename T> constexpr auto type_name() {...}

            auto get_mix()
            {
                return std::make_tuple(42U, 42, 3.14, 'b', "Hello");
            }

            int main(){
                auto [uns, n, pi, letter, label] = get_mix();
                std::cout << "[" << type_name<decltype(uns)>() << ", "
                          << type_name<decltype(n)>() << ", "
                          << type_name<decltype(pi)>() << ", "
                          << type_name<decltype(letter)>() << ", "
                          << type_name<decltype(label)>() << "]\n";
            }
        \end{Cpp}
        \smallskip
        \begin{Bash}[numbers=none]
            |+[unsigned int, int, double, char, const char *]+|
        \end{Bash}
    \end{varblock}
    \begin{varblock}{example}[\textwidth]{Even cooler}<only@2>
        \begin{Cpp}
            #include |+<iostream>+|
            #include |+<unordered_map>+|

            int main(){
                std::unordered_map<std::string, int> mapping {
                    {"a", 1},
                    {"b", 2},
                    {"c", 3}
                };

                // Destructure by reference.
                for (const auto& [key, content] : mapping) {
                    std::cout << key << " -> " << content << "\n";
                }
            }
        \end{Cpp}
        \smallskip
        \begin{Bash}[numbers=none]
            |+b -> 2
            c -> 3
            a -> 1+|
        \end{Bash}
    \end{varblock}
\end{frame}
%============================================%

%============================================%
\section{Selection statements with initializer}
%~~~~~~~~~~~~~~~~~~~~~~~~~~~~~~~~~~~~~~~~~~~~%
\begin{frame}[fragile]{\insertsectionhead}
    \vspace{-3mm}
    \begin{itemize}
        \item New versions of the \CPP|if| and \CPP|switch| statements which simplify common code patterns and help keeping scopes tight.\\
              \begin{varblock}{}[0.73\textwidth]{How they look like:}
                  \small
                  \CPP|if(init-statement; condition)|\\[1mm]
                  \CPP|switch(init-statement; condition)|
              \end{varblock}
              \par\medskip
        \item Names declared by the \CPP|init-statement| (if it is a declaration) and names declared by \CPP|condition| (if it is a declaration) are in the same scope of all branches.
    \end{itemize}
\end{frame}
%~~~~~~~~~~~~~~~~~~~~~~~~~~~~~~~~~~~~~~~~~~~~%
\begin{frame}[fragile]{}
    \begin{varblock}{example}[\textwidth]{New if and switch feature}<only@1>
        \begin{Cpp}
            #include |+<iostream>+|
            #include |+<map>+|
            #include |+<iomanip>+|
            #include |+"widget.hpp"+|

            int main(int argc, char *argv[])
            {
                std::map<int, std::string> m;
                if (auto it = m.find(10); it != m.end())
                    std::cout << it->second.size();
                else
                    std::cout << std::boolalpha << (it==m.end()) << "\n";

                switch (Widget gadget(argc, argv);
                        auto s = gadget.status())
                {
                    case OK: gadget.zip(); break;
                    case Bad: throw BadWidget(s.message());
                }
            }
        \end{Cpp}
        \begin{Bash}[numbers=none]
            |+true+|
        \end{Bash}
    \end{varblock}
\end{frame}
%============================================%

%============================================%
\section{Compile time if conditions}
%~~~~~~~~~~~~~~~~~~~~~~~~~~~~~~~~~~~~~~~~~~~~%
\begin{frame}[fragile]{\insertsectionhead}
    \vspace{-3mm}
    \begin{itemize}
        \item In a \CPP|constexpr if| statement, the value of the condition must be a \URL[PB]{https://en.cppreference.com/w/cpp/language/constant_expression\#Converted_constant_expression}{contextually converted constant expression of type \CPP|bool|} \Remark{until C++23}
        \item  Depending on the condition, either the \CPP|if| or the \CPP|else| statement is discarded.
        \item The return statements in a discarded statement do not participate in function return type deduction
        \item Outside a template, a discarded statement is fully checked\\
              {\footnotesize$\to\;$\CPP|if constexpr| is not a substitute for the \CPP|#if| preprocessing directive}
        \item More information: \URL[PB]{https://en.cppreference.com/w/cpp/language/if\#Constexpr_if}{C++ reference}
    \end{itemize}
\end{frame}
%~~~~~~~~~~~~~~~~~~~~~~~~~~~~~~~~~~~~~~~~~~~~%
\begin{frame}[fragile]{}
    \begin{varblock}{example}[\textwidth]{Compile time if clause}<only@1>
        \begin{Cpp}
            template <typename T>
            constexpr bool isIntegral() {
                if constexpr (std::is_integral<T>::value) {
                    return true;
                } else {
                    return false;
                }
            }
            struct Widget {};
            int main(){
                if constexpr(false) {
                    int n = 0;
                    //int *p = n; // Error even though discarded
                }
                static_assert(isIntegral<char>() == true);
                static_assert(isIntegral<double>() == false);
                static_assert(isIntegral<Widget>() == true);
            }
        \end{Cpp}
        \begin{Bash}[numbers=none]
            |+$ clang++ -std=c++17 Example.cpp -o Example
            Constexpr-if.cpp:26:5: error: static_assert failed due to requirement 'isIntegral<Widget>() == true'
            [...]+|
        \end{Bash}
    \end{varblock}
\end{frame}
%============================================%

%============================================%
\section{Inline variables}
%~~~~~~~~~~~~~~~~~~~~~~~~~~~~~~~~~~~~~~~~~~~~%
\begin{frame}[fragile]{\insertsectionhead}
    \vspace{-3mm}
    \begin{itemize}
        \item An inline variable has the same semantics as an inline function.\\[1mm]
              {\scriptsize
                Together with other properties there may be more than one definition of an inline variable in the program as long as each definition appears in a different translation unit and (for non-static inline variables) all definitions are identical.
                For example, an inline variable may be defined in a header file that is included in multiple source files.\par}
        \item More information: \URL[PB]{https://en.cppreference.com/w/cpp/language/inline}{C++ reference}
    \end{itemize}
    \begin{varblock}{example}[\textwidth]{Define a static member variable in \textbf{header} file!}<only@1>
        \begin{Cpp}
            struct Widget {
                Widget() : id{count++} {}
                ~Widget() { count--; }
                int id;
                // declare and initialize count to 0 within the class
                static inline int count{0};
            };
        \end{Cpp}
    \end{varblock}
\end{frame}
%============================================%

%============================================%
\section{New rules for auto deduction from braced-init-list}
%~~~~~~~~~~~~~~~~~~~~~~~~~~~~~~~~~~~~~~~~~~~~%
\begin{frame}[fragile]{\insertsectionhead}
    \begin{itemize}
        \item Till C++14$^\star$, in
              \begin{center}
                  \CPP|auto x{3};|
              \end{center}
              \CPP|x| is deduced to be a
              \begin{center}
                  \CPP|std::initializer_list<int>|
              \end{center}
              which is at least unexpected \Remark{if not misleading}
        \item In C++17 it is now deduced to be an \CPP|int|
        \item Direct-list-initialization from a multiple-element braced-init-list is now ill-formed
    \end{itemize}
    \PrepareURLsymbol[PB]
    \FrameRemark{$^\star$More precisely till paper \URL*{http://www.open-std.org/jtc1/sc22/wg21/docs/papers/2014/n3922.html}{N3922} approved in 2014 already in Urbana by the core working group (CWG) in \URL*{https://www.open-std.org/jtc1/sc22/wg21/docs/papers/2014/n4251.html}{motion 16}.}
\end{frame}
%~~~~~~~~~~~~~~~~~~~~~~~~~~~~~~~~~~~~~~~~~~~~%
\begin{frame}[fragile]{}
    \PrepareURLsymbol[red]
    \begin{varblock}{example}[\textwidth]{Pretty logic and easy to remember now}<only@1>
        \begin{Cpp}
            #include |+<iostream>+|
            #include |+<utility>+|
            #include |+<string_view>+| // @|\URL*[red]{https://stackoverflow.com/a/56766138}{Very cool idea}|@
            template <typename T> constexpr auto type_name() {...}

            int main(){
                //auto x1 {1, 2, 3}; // error: not a single element
                auto x2 = {1, 2, 3};
                auto x3 {3u};
                auto x4 {3.0f};
                std::cout << type_name<decltype(x2)>() << "\n"
                          << type_name<decltype(x3)>() << "\n"
                          << type_name<decltype(x4)>() << "\n";
            }
        \end{Cpp}
        \begin{Bash}[numbers=none]
            |+std::initializer_list<int>
            unsigned int
            float+|
        \end{Bash}
    \end{varblock}
\end{frame}
%============================================%

%============================================%
\section{Nested namespaces}
%~~~~~~~~~~~~~~~~~~~~~~~~~~~~~~~~~~~~~~~~~~~~%
\begin{frame}[fragile]{\insertsectionhead}
    \vspace{-6mm}
    \begin{varblock}{example}[\textwidth]{Reducing boilerplate}<only@1>
        \begin{Cpp}
            // No alternatives till C++14
            namespace ns1 {
                namespace ns2 {
                    namespace ns3 {
                        int i;
                    }
                }
            }

            // The code above can be written like this with C++17
            namespace ns1::ns2::ns3 {
                int i;
            }
        \end{Cpp}
    \end{varblock}
\end{frame}
%============================================%

%============================================%
\section{Direct-list-initialization of enums}
%~~~~~~~~~~~~~~~~~~~~~~~~~~~~~~~~~~~~~~~~~~~~%
\begin{frame}[fragile]{\insertsectionhead}
    \vspace{-3mm}
    An enumeration can be initialized from an integer without a cast, using list initialization, if all of the following are true:
    \begin{itemize}
        \small
        \item the initialization is direct-list-initialization
        \item the initializer list has only a single element
        \item the enumeration is either scoped or unscoped with underlying type fixed
        \item the conversion is non-narrowing
    \end{itemize}
    \begin{varblock}{example}[\textwidth]{}<only@1>
        \begin{Cpp}
            enum byte : unsigned char {};
            byte b1 {0};         // OK
            byte b2 {-1};        // ERROR: narrowing conversion
            byte b3 = byte{1};   // OK
            byte b4 = byte{256}; // ERROR: narrowing conversion

            enum Type {type_A, type_B};
            Type x{1};           // ERROR: underline type not fixed
        \end{Cpp}
    \end{varblock}
\end{frame}
%============================================%

%============================================%
\section{New attributes}
%~~~~~~~~~~~~~~~~~~~~~~~~~~~~~~~~~~~~~~~~~~~~%
\begin{frame}[fragile]{\insertsectionhead}
    \vspace{-3mm}
    \begin{tabular}{p{\textwidth}}
        \CPP|[[fallthrough]]|
             \URL[PB]{https://en.cppreference.com/w/cpp/language/attributes/fallthrough}{C++ reference}\\[1mm]
             It indicates to the compiler that falling through in a \CPP|switch| statement is intended behaviour\\[3mm]
        \CPP|[[nodiscard]]|
             \URL[PB]{https://en.cppreference.com/w/cpp/language/attributes/nodiscard}{C++ reference}\\[1mm]
             It encourages the compiler to issue a warning if a function declared nodiscard or a function returning an enumeration or class declared nodiscard by value is called from a discarded-value expression other than a cast to \CPP|void|\\[3mm]
        \CPP|[[maybe_unused]]|
             \URL[PB]{https://en.cppreference.com/w/cpp/language/attributes/maybe_unused}{C++ reference}\\[1mm]
             It suppresses compiler warnings on unused entities\\
    \end{tabular}
\end{frame}
%~~~~~~~~~~~~~~~~~~~~~~~~~~~~~~~~~~~~~~~~~~~~%
\begin{frame}[fragile]{}
    \begin{varblock}{example}[\textwidth]{Fall through cases}<only@1>
        \begin{Cpp}
            switch (n) {
                case 1:
                case 2:
                    foo();
                    [[fallthrough]];
                case 3: // no warning on fallthrough
                    bar();
                case 4: // compiler may warn on fallthrough
                    if(n<3)
                    {
                        foo();
                        [[fallthrough]];
                    }
                    else
                        return;
                default:
                    throw std::logic_error("Hit default!");
            }
        \end{Cpp}
    \end{varblock}
    \begin{varblock}{example}[\textwidth]{Warn when discarding nodiscard types returned by value}<only@2>
        \begin{Cpp}
            // Only issues a warning when returned by value.
            struct [[nodiscard]] error_info {
                int id{0};
            };

            error_info do_something() {
                return error_info{};
            }

            error_info& do_something_ref() {
                static error_info err{};
                return err;
            }

            int main(){
                do_something();      // Warning
                do_something_ref();  // No warning!
            }
        \end{Cpp}
        \begin{Bash}[numbers=none]
            |+Attributes.cpp:16:5: warning: ignoring return value of function declared with 'nodiscard' attribute [...]+|
        \end{Bash}
    \end{varblock}
    \begin{onlyenv}<3>
        \begin{varblock}{example}[\textwidth]{Warn when discarding return values}
            \begin{Cpp}
                [[nodiscard]] bool do_something() {
                    return is_success();
                }

                do_something(); // Warning: ignoring return value
                                // declared with attribute 'nodiscard'
            \end{Cpp}
        \end{varblock}
        \bigskip
        \begin{varblock}{example}[\textwidth]{Unused entities}
            \begin{Cpp}
                void my_callback(std::string message,
                                 [[maybe_unused]] bool is_error) {
                    // Don't care if message is an error message
                    log(message);
                }
            \end{Cpp}
        \end{varblock}
    \end{onlyenv}
\end{frame}
%============================================%

%============================================%
\section{Mandatory elision of copy/move operations}
%~~~~~~~~~~~~~~~~~~~~~~~~~~~~~~~~~~~~~~~~~~~~%
\begin{frame}{\insertsectionhead}
    In C++17, often$^{\star}$, in return and assignment statements:\hfill {\footnotesize\URL[PB]{https://en.cppreference.com/w/cpp/language/copy_elision}{[Precise rule]}}
    \begin{itemize}
        \item The compilers \alert{\textbf{are required}} to omit the copy and move construction of class objects
        \item Even if the copy/move constructor and the destructor have observable side-effects
        \item The objects are constructed directly into the storage where they would otherwise be copied/moved to \item The copy/move constructors need not be present or accessible
    \end{itemize}
    \begin{varblock}{alert}[\textwidth]{When copy/move operations are not guaranteed}
        Even if the copy/move constructor is not called, it still must be present and accessible, otherwise the program is ill-formed!
    \end{varblock}
    \PrepareURLsymbol[PB]
    \FrameRemark{$^{\star}$ \textbf{often} $\equiv$ when the operand or initializer expression is a \URL*{https://stackoverflow.com/q/3601602/14967071}{prvalue} of the same class type (ignoring cv-qualification).}
\end{frame}
%~~~~~~~~~~~~~~~~~~~~~~~~~~~~~~~~~~~~~~~~~~~~%
\begin{frame}[fragile]{}
    \begin{varblock}{example}[\textwidth]{}<only@1>
        \begin{Cpp}
            #include |+<iostream>+|

            struct C {
                C() { std::cout << "Default ctor called.\n"; }
                C(const C&) { std::cout << "Copy ctor called.\n"; }
                C(C&&) { std::cout << "Move ctor called.\n"; }
            };

            C foo() { return C(); } // Guaranteed to perform copy elision
            C bar() { C c; return c; } // Maybe performs copy elision

            int main() {
                C obj = foo(); // Move constructor isn't called
                std::cout << "-------\n";
                C obj2 = bar();
            }
        \end{Cpp}
        \begin{lstlisting}[style=MyBash, numbers=none]
            |+$ g++ -fno-elide-constructors -std=c++17\
                  -o Copy-move-elision Copy-move-elision.cpp\
                  && ./Copy-move-elision
            Default ctor called.
            -------
            Default ctor called.
            Move ctor called.+|
        \end{lstlisting}
    \end{varblock}
    \begin{varblock}{example}[\textwidth]{}<only@2>
        \begin{Cpp}
            #include |+<iostream>+|

            struct C {
                C() { std::cout << "Default ctor called.\n"; }
                C(const C&) = delete;
                C(C&&) = delete;
            };

            C foo() { return C(); } // Guaranteed to perform copy elision

            /*
             * Uncommenting results in compilation error, since
             * move/copy elision is not guaranteed here!
             */
            // C bar() { C c; return c; }

            int main() {
                C obj = foo(); //Move constructor isn't called
            }
        \end{Cpp}
        \begin{lstlisting}[style=MyBash, numbers=none]
            |+$ g++ -fno-elide-constructors -std=c++17\
                  -o Copy-move-elision Copy-move-elision.cpp\
                  && ./Copy-move-elision
            Default ctor called.+|
        \end{lstlisting}
    \end{varblock}
    \begin{varblock}{example}[\textwidth]{}<only@3>
        \begin{Cpp}
            #include |+<iostream>+|

            struct C {
                C() { std::cout << "Default ctor called.\n"; }
                C(const C&) = delete;
                C(C&&) = delete;
            };

            C bar() { C c; return c; }

            int main() { return 0; }
        \end{Cpp}
        \begin{lstlisting}[style=MyBash, numbers=none]
            |+$ g++ -std=c++17\
                  -o Copy-move-elision Copy-move-elision.cpp\
                  && ./Copy-move-elision
            Copy-move-elision.cpp:9:23: error: call to deleted constructor of 'C'
            C bar() { C c; return c; }
            ^
            Copy-move-elision.cpp:5:5: note: 'C' has been explicitly marked deleted here
            C(const C&) = delete;
            ^
            1 error generated.+|
        \end{lstlisting}
    \end{varblock}
\end{frame}
%============================================%


    %~~~~~~~~~~~~~~~~~~~~~~~~~~~~~~~~~~~~~~~~~~~~%
    \label{P2}
    \part{New library features}
    %-------------------------------%
%  Author: Alessandro Sciarra   %
%    Date: 15 Jun 2022          %
%-------------------------------%

%~~~~~~~~~~~~~~~~~~~~~~~~~~~~~~~~~~~~~~~~~~~~%
\begin{frame}[fragile]{C++17 new library features I will \textbf{not} cover}
    \begin{tabular}{l}
        \URL[PB]{https://en.cppreference.com/w/cpp/types/byte}{~std::byte}\\[1mm]
        \URL[PB]{https://en.cppreference.com/w/cpp/utility/functional/invoke}{~std::invoke}\Remark{More understandable \URL[PB]{https://stackoverflow.com/q/43680182/14967071}{SO question}}\\[1mm]
        \URL[PB]{https://en.cppreference.com/w/cpp/utility/apply}{~std::apply} $\;\to\;$ {\footnotesize\URL[PB]{https://stackoverflow.com/q/52449163/14967071}{What's the difference with \CPP|std::invoke|\,?}}\\[1mm]
        \URL[PB]{https://wg21.link/P0083R3}{~Splicing maps and sets} \Remark{i.e. effectively transfer elements \`a la \CPP|std::list::splice|}

    \end{tabular}
\end{frame}
%~~~~~~~~~~~~~~~~~~~~~~~~~~~~~~~~~~~~~~~~~~~~%
\begin{frame}{C++17 new library features I will cover}
    \tableofcontents
\end{frame}
%============================================%

%============================================%
\section{\CPP|std::variant|}
%~~~~~~~~~~~~~~~~~~~~~~~~~~~~~~~~~~~~~~~~~~~~%
\begin{frame}[fragile]{\insertsectionhead}
    \vspace{-3mm}
    \begin{itemize}
        \item More information: \URL[PB]{}{C++ reference}
    \end{itemize}
\end{frame}
%~~~~~~~~~~~~~~~~~~~~~~~~~~~~~~~~~~~~~~~~~~~~%
\begin{frame}[fragile]{}
    \begin{varblock}{example}[\textwidth]{}<only@1>
        \begin{Cpp}
        \end{Cpp}
        \begin{Bash}[numbers=none]
            |++|
        \end{Bash}
    \end{varblock}
\end{frame}
%============================================%

%============================================%
\section{\CPP|std::optional|}
%~~~~~~~~~~~~~~~~~~~~~~~~~~~~~~~~~~~~~~~~~~~~%
\begin{frame}[fragile]{\insertsectionhead}
    \vspace{-3mm}
    \begin{itemize}
        \item More information: \URL[PB]{}{C++ reference}
    \end{itemize}
\end{frame}
%~~~~~~~~~~~~~~~~~~~~~~~~~~~~~~~~~~~~~~~~~~~~%
\begin{frame}[fragile]{}
    \begin{varblock}{example}[\textwidth]{}<only@1>
        \begin{Cpp}
        \end{Cpp}
        \begin{Bash}[numbers=none]
            |++|
        \end{Bash}
    \end{varblock}
\end{frame}
%============================================%

%============================================%
\section{\CPP|std::any|}
%~~~~~~~~~~~~~~~~~~~~~~~~~~~~~~~~~~~~~~~~~~~~%
\begin{frame}[fragile]{\insertsectionhead}
    \vspace{-3mm}
    \begin{itemize}
        \item More information: \URL[PB]{}{C++ reference}
    \end{itemize}
\end{frame}
%~~~~~~~~~~~~~~~~~~~~~~~~~~~~~~~~~~~~~~~~~~~~%
\begin{frame}[fragile]{}
    \begin{varblock}{example}[\textwidth]{}<only@1>
        \begin{Cpp}
        \end{Cpp}
        \begin{Bash}[numbers=none]
            |++|
        \end{Bash}
    \end{varblock}
\end{frame}
%============================================%

%============================================%
\section{\CPP|std::string\_view|}
%~~~~~~~~~~~~~~~~~~~~~~~~~~~~~~~~~~~~~~~~~~~~%
\begin{frame}[fragile]{\insertsectionhead}
    \vspace{-3mm}
    \begin{itemize}
        \item More information: \URL[PB]{}{C++ reference}
    \end{itemize}
\end{frame}
%~~~~~~~~~~~~~~~~~~~~~~~~~~~~~~~~~~~~~~~~~~~~%
\begin{frame}[fragile]{}
    \begin{varblock}{example}[\textwidth]{}<only@1>
        \begin{Cpp}
        \end{Cpp}
        \begin{Bash}[numbers=none]
            |++|
        \end{Bash}
    \end{varblock}
\end{frame}
%============================================%

%============================================%
\section{\CPP|std::filesystem|}
%~~~~~~~~~~~~~~~~~~~~~~~~~~~~~~~~~~~~~~~~~~~~%
\begin{frame}[fragile]{\insertsectionhead}
    \vspace{-3mm}
    \begin{itemize}
        \item More information: \URL[PB]{}{C++ reference}
    \end{itemize}
\end{frame}
%~~~~~~~~~~~~~~~~~~~~~~~~~~~~~~~~~~~~~~~~~~~~%
\begin{frame}[fragile]{}
    \begin{varblock}{example}[\textwidth]{}<only@1>
        \begin{Cpp}
        \end{Cpp}
        \begin{Bash}[numbers=none]
            |++|
        \end{Bash}
    \end{varblock}
\end{frame}
%============================================%

%============================================%
\section{Parallel algorithms}
%~~~~~~~~~~~~~~~~~~~~~~~~~~~~~~~~~~~~~~~~~~~~%
\begin{frame}[fragile]{\insertsectionhead}
    \vspace{-3mm}
    \begin{itemize}
        \item More information: \URL[PB]{}{C++ reference}
    \end{itemize}
\end{frame}
%~~~~~~~~~~~~~~~~~~~~~~~~~~~~~~~~~~~~~~~~~~~~%
\begin{frame}[fragile]{}
    \begin{varblock}{example}[\textwidth]{}<only@1>
        \begin{Cpp}
        \end{Cpp}
        \begin{Bash}[numbers=none]
            |++|
        \end{Bash}
    \end{varblock}
\end{frame}
%============================================%



    %~~~~~~~~~~~~~~~~~~~~~~~~~~~~~~~~~~~~~~~~~~~~%
    \part{Your next C++ years}
    \begin{frame}{\uncover<2->{Which is the best way to go, now?}}
        \begin{onlyenv}<1>
            \begin{tikzpicture}[remember picture, overlay]
                \node at (current page.center) {
                    \includegraphics[width=0.88\textwidth]{LearnCpp}
                };
            \end{tikzpicture}
        \end{onlyenv}
        \vspace{5mm}
        \begin{overlayarea}{\textwidth}{0.23\textheight}
            \begin{onlyenv}<2>
                \begin{itemize}
                    \item Consider exploring new features
                    \item Take a decision in your codebase/team
                    \item There are better and better way to write the same code!
                \end{itemize}
            \end{onlyenv}
            \begin{onlyenv}<3>
                \begin{center}
                    \LARGE Thank you! Questions?
                \end{center}
            \end{onlyenv}
        \end{overlayarea}
        \begin{uncoverenv}<2->
            \begin{center}
                \begin{tikzpicture}[every node/.style={inner sep=0pt}, pgfornamentstyle/.style={color=PS}]
                    \def\ornsize{11mm}
                    \node[font=\large, text=PB](Text) {%
                        \begin{tabular}{c}
                            Every day, work to refine the skills you have\\
                            and to add new tools to your repertoire.\\[1mm]
                            {\normalsize---~The pragmatic programmer~---}
                        \end{tabular}
                    };
                    \node[shift={(-12mm,6mm)},anchor=north west] (CNW)
                    at (Text.north west) {\pgfornament[width=\ornsize]{61}};
                    \node[shift={(12mm,6mm)},anchor=north east] (CNE)
                    at (Text.north east) {\pgfornament[width=\ornsize,symmetry=v]{61}};
                    \node[shift={(-12mm,-5mm)},anchor=south west] (CSW)
                    at (Text.south west) {\pgfornament[width=\ornsize,symmetry=h]{61}};
                    \node[shift={(12mm,-5mm)},anchor=south east] (CSE)
                    at (Text.south east) {\pgfornament[width=\ornsize,symmetry=c]{61}};
                    \pgfornamentline{CNW.north east}{CNE.north west}{3}{88}
                    \pgfornamentline{CSW.south east}{CSE.south west}{3}{88}
                    \pgfornamentline{CNW.south west}{CSW.north west}{2}{88}
                    \pgfornamentline{CNE.south east}{CSE.north east}{2}{88}
                \end{tikzpicture}
            \end{center}
        \end{uncoverenv}
    \end{frame}
    %~~~~~~~~~~~~~~~~~~~~~~~~~~~~~~~~~~~~~~~~~~~~%
\end{document}